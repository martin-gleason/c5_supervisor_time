\documentclass[]{article}
\usepackage{lmodern}
\usepackage{amssymb,amsmath}
\usepackage{ifxetex,ifluatex}
\usepackage{fixltx2e} % provides \textsubscript
\ifnum 0\ifxetex 1\fi\ifluatex 1\fi=0 % if pdftex
  \usepackage[T1]{fontenc}
  \usepackage[utf8]{inputenc}
\else % if luatex or xelatex
  \ifxetex
    \usepackage{mathspec}
  \else
    \usepackage{fontspec}
  \fi
  \defaultfontfeatures{Ligatures=TeX,Scale=MatchLowercase}
\fi
% use upquote if available, for straight quotes in verbatim environments
\IfFileExists{upquote.sty}{\usepackage{upquote}}{}
% use microtype if available
\IfFileExists{microtype.sty}{%
\usepackage{microtype}
\UseMicrotypeSet[protrusion]{basicmath} % disable protrusion for tt fonts
}{}
\usepackage[margin=1in]{geometry}
\usepackage{hyperref}
\hypersetup{unicode=true,
            pdftitle={Supervisor Timesheets},
            pdfborder={0 0 0},
            breaklinks=true}
\urlstyle{same}  % don't use monospace font for urls
\usepackage{graphicx,grffile}
\makeatletter
\def\maxwidth{\ifdim\Gin@nat@width>\linewidth\linewidth\else\Gin@nat@width\fi}
\def\maxheight{\ifdim\Gin@nat@height>\textheight\textheight\else\Gin@nat@height\fi}
\makeatother
% Scale images if necessary, so that they will not overflow the page
% margins by default, and it is still possible to overwrite the defaults
% using explicit options in \includegraphics[width, height, ...]{}
\setkeys{Gin}{width=\maxwidth,height=\maxheight,keepaspectratio}
\IfFileExists{parskip.sty}{%
\usepackage{parskip}
}{% else
\setlength{\parindent}{0pt}
\setlength{\parskip}{6pt plus 2pt minus 1pt}
}
\setlength{\emergencystretch}{3em}  % prevent overfull lines
\providecommand{\tightlist}{%
  \setlength{\itemsep}{0pt}\setlength{\parskip}{0pt}}
\setcounter{secnumdepth}{0}
% Redefines (sub)paragraphs to behave more like sections
\ifx\paragraph\undefined\else
\let\oldparagraph\paragraph
\renewcommand{\paragraph}[1]{\oldparagraph{#1}\mbox{}}
\fi
\ifx\subparagraph\undefined\else
\let\oldsubparagraph\subparagraph
\renewcommand{\subparagraph}[1]{\oldsubparagraph{#1}\mbox{}}
\fi

%%% Use protect on footnotes to avoid problems with footnotes in titles
\let\rmarkdownfootnote\footnote%
\def\footnote{\protect\rmarkdownfootnote}

%%% Change title format to be more compact
\usepackage{titling}

% Create subtitle command for use in maketitle
\newcommand{\subtitle}[1]{
  \posttitle{
    \begin{center}\large#1\end{center}
    }
}

\setlength{\droptitle}{-2em}
  \title{Supervisor Timesheets}
  \pretitle{\vspace{\droptitle}\centering\huge}
  \posttitle{\par}
  \author{}
  \preauthor{}\postauthor{}
  \date{}
  \predate{}\postdate{}

\usepackage{booktabs}
\usepackage{longtable}
\usepackage{array}
\usepackage{multirow}
\usepackage[table]{xcolor}
\usepackage{wrapfig}
\usepackage{float}
\usepackage{colortbl}
\usepackage{pdflscape}
\usepackage{tabu}
\usepackage{threeparttable}
\usepackage{threeparttablex}
\usepackage[normalem]{ulem}
\usepackage{makecell}

\begin{document}
\maketitle

The following PDF is an example of the work that can be done with R to
automate the reporting of the \emph{c5 Supervisor} project.

When a member of the Steering committee inputs their task, it updates
the Google spreadsheet. Then, this report works behind the scenes to
access the spreadsheet and find insights that are otherwise hidden. For
example, the following graph shows how many tasks were inputted by
members of the committee:

\includegraphics{reports_files/figure-latex/hours_worked-1.pdf}

Automating the report programatically does not just cut down on errors
by eliminating the cut/copy-paste-from-a-spreadsheet-to-a-word-document
step. It also allows for customization. First and foremost, the graph's
labels can be edited, and the color scheme can be set to a color-blind
friendly scheme. Additionally, this report can fix DCPO Spooner's name
so she is attributed accordingly.

All of the names in the document stem from the email addresses used in
the form. The reason is simple: Most emails are
\texttt{firstname.lastname@cookcountyil.gov}. This allows for a simple
function to split the first and last name. DCPO Spooher, however,
presents a challenge as her email is
\texttt{melissa.parise@cookcountyil.gov}. Fixing this would be difficult
in the spreadsheet, but given the nature of this report, it is a trivial
task.

In addition to graphs, tables can also be added: \newpage
\rowcolors{2}{gray!6}{white}

\begin{table}

\caption{\label{tab:tasks_count}Person Hours: C5 Project April 2018 - May 2018}
\centering
\begin{tabular}[t]{l|l|r|r}
\hiderowcolors
\hline
first\_name & last\_name & Tasks\_entered & total\_hours\\
\hline
\showrowcolors
Kevin & Hickey & 4 & 5\\
\hline
Martin & Gleason & 3 & 7\\
\hline
Melissa & Spooner & 1 & 5\\
\hline
Nicole & Paryz & 1 & 2\\
\hline
Nicole & Williams & 3 & 5\\
\hline
Richard & Naujokas & 7 & 22\\
\hline
Tamar & Stockley & 19 & 50\\
\hline
\end{tabular}
\end{table}

\rowcolors{2}{white}{white} This summarizes the tasks accomplished
quickly.

\begin{table}

\caption{\label{tab:person_hours}Total Tasks and Hours}
\centering
\begin{tabular}[t]{r|r}
\hline
Tasks Entered &  Total Hours\\
\hline
38 & 96\\
\hline
\end{tabular}
\end{table}

The above table can also be cited within the text. For instance, the
total number of tasks are 38 and the total hours worked on the project
to date is 96.

Lastly, the tables could be grouped by date. \rowcolors{2}{pink}{white}

\begin{table}[!h]

\caption{\label{tab:table_by_dates}Task per Month}
\centering
\begin{tabular}[t]{l|l|r|r}
\hiderowcolors
\hline
last\_name & Month Completed & Number of Projects & Total Hours Per Month\\
\hline
\showrowcolors
Gleason & May & 3 & 7\\
\hline
Hickey & Mar & 1 & 2\\
\hline
Hickey & Apr & 2 & 2\\
\hline
Hickey & May & 1 & 1\\
\hline
Naujokas & Apr & 1 & 14\\
\hline
Naujokas & May & 6 & 8\\
\hline
Paryz & Mar & 1 & 2\\
\hline
Spooner & Mar & 1 & 5\\
\hline
Stockley & Mar & 7 & 13\\
\hline
Stockley & Apr & 12 & 37\\
\hline
Williams & Mar & 1 & 2\\
\hline
Williams & Apr & 1 & 1\\
\hline
Williams & May & 1 & 2\\
\hline
\end{tabular}
\end{table}

\rowcolors{2}{white}{white} This would allow for quick, easy, and
repeateable reporting with a few extra lines of code, all without having
to cut and paste between Excel/Google Sheets and Word.


\end{document}
